\documentclass[12pt,twoside]{article}
\usepackage{jmlda}

%\NOREVIEWERNOTES
\title
    [Исследование свойств локальных моделей при пространственном декодировании сигналов головного мозга] 
    {Исследование свойств локальных моделей при пространственном декодировании сигналов головного мозга}
\author
    [Самохина~А.\,М.] % список авторов для колонтитула; не нужен, если основной список влезает в колонтитул
    {Самохина~А.\,М.$^1$, Болоболова~Н.\,А.$^1$, Шиянов~В.\,A.$^1$} % основной список авторов, выводимый в оглавление
    %[Автор~И.\,О.$^1$, Соавтор~И.\,О.$^2$, Фамилия~И.\,О.$^2$] % список авторов, выводимый в заголовок; не нужен, если он не отличается от основного
\thanks
    {
   Научный руководитель:  Стрижов~В.\,В.
   Задачу поставил:  Стрижов~В.\,В.
    Консультант:  Исаченко~Р.}
\email
    {alina.samokhina@phystech.edu}
\organization
    {$^1$Московский физико-технический институт(ГУ)}
\abstract
    { {\textbf{Аннотация:}} Данная статья посвящена методам прогнозирования движения с помощью сигналов электрокортикограммы головного мозга.  Основная задача исследования~--- показать, что наблюдаемое изменение зон активности мозга является информативным признаком для построения нейрокомпьютерного интерфейса. Чтобы проследить связь между сигналами мозга и движением, в данном исследовании рассматриваются различные методы генерации признаков. В частности, снижение размерности исходного признакового пространства с помощью локальных моделей. За новое пространство признаков берётся пространство параметров модели. Предлагаемое признаковое пространство позволяет обоснованно использовать данные электрокортикограмм при построении моделей нейрокомпьютерного взаимодействия.

\bigskip
\textbf{Ключевые слова}: \emph {электрокортикограмма, нейрокомпьютерный интерфейс}.}

\titleEng
    {Research on the properties of local models in spatial decoding of the brain signals}
\authorEng
    {Samokhina~A.\,M.$^1$, Bolobolova~N.\,A.$^1$, Shiyanov~V.\,A.$^1$}
\organizationEng
    {$^1$ Moscow institute of Physics and Technology (SU)}
\abstractEng
    { \textbf{Abstract:} This paper is devoted to the methods of ECoG signal processing and predicting motion using the results. The main purpose of the research is to show that changes in areas of brain activity is an informative feature for BCI modelling. To see the link between brain signals and motion, we look at different types of feature engineering and compare them. The new  feature space will allow to use ECoG data for building BCI, using ECoG data.

    \bigskip
    \textbf{Keywords}: \emph{ECoG, BCI}.}
    
\begin{document}
\maketitle
\bigskip
\bigskip
\bigskip
\bigskip
\maketitleSecondary
%\linenumbers

\section{Введение}
Нейрокомпьютерный интерфейс позволяет декодировать активность головного мозга для испоьзования внешними устройствами. Это позволяет создавать механизированные конечности, управляемые сигналами головного мозга \cite{Donoghue2008}. В связи с этим в последнее время большое количество работ посвящено методам считывания мозговой активности и декодирования полученной информации \cite{Hu2018}\cite{Song2017}\cite{Loza2017}\cite{Eliseyev2016}\cite{Gaglianese2016}\cite{Bundy2016}\cite{Morishita2014}, и другие. 
В этой работе мы использовали данные сигналов, полученных инвазивным методом электрокортикографии \cite{Sirven2014}. Проблема декодирования этих сигналов заключается в избыточной размерности сигнала: в данном случае модель является неустойчивой. Построение систем нейрокомпьютерного интерфейса подразумевает использование моделей простых и устойчивых. Таким образом, важным этапом построения модели является построение адекватного признакового пространства. 
Стандартные подходы состоят в извлечении информативных признаков из пространственных, частотных и временных данных\cite{Morishita2014}\cite{Alexander2013}. Большинство методов исследуют только некоторую часть данных, чаще всего, исследуют частотные характеристики сигналов\cite{Chin2007}. В недавнее время стали появляться подходы \cite{Eliseyev2016}\cite{Motrenko2018}, позволяющие рассматривать все признаки вне зависимости от их природы. Для построения признакового пространства используется большое количество методов.  Наиболее распространёнными считаются алгоритм PLS\cite{Rosipal2006}\cite{Eliseyev2016}, PCA\cite{Zhao2010}. В работе \cite{Zhao2014} используются алгоритмы, построенные на скрытых марковских моделях, а в  работах \cite{Loza2017}\cite{Song2017} авторы рассматривают различные участки сигнала как "слова". 

В данной работе предлагается использовать локальную модель. Это позволило значительно сократить количество признаков, сохранив структуру данных. Движение фронта возбуждения  приближено несколькими векторами и локальной моделью. Параметры модели используются в качестве признаков. Таким образом, сформировано признаковое пространство и локально смоделирован пространственный сигнал, на основе которого можно построить устойчивую, простую и адекватную прогностическую модель.


Полученный  метод, что значительно снижает размерность данных,  использует пространственную информацию и принимает во внимание свойства распространения сигнала.во избежание смещения модели мы были вынуждены наложить строгое ограничение на вид при выборе семейства локальных моделей.

\bibliographystyle{plain}
\bibliography{Samokhina2018Project17}


\end{document}
