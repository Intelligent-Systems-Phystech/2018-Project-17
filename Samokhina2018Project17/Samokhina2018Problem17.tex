\documentclass[12pt,twoside]{article}
\usepackage{jmlda}

%\NOREVIEWERNOTES
\title
    [Исследование свойств локальных моделей при пространственном декодировании сигналов головного мозга] 
    {Исследование свойств локальных моделей при пространственном декодировании сигналов головного мозга}
\author
    [Самохина~А.\,М.] % список авторов для колонтитула; не нужен, если основной список влезает в колонтитул
    {Самохина~А.\,М.$^1$, Болоболова~Н.\,А.$^1$, Шиянов~В.\,A.$^1$} % основной список авторов, выводимый в оглавление
    %[Автор~И.\,О.$^1$, Соавтор~И.\,О.$^2$, Фамилия~И.\,О.$^2$] % список авторов, выводимый в заголовок; не нужен, если он не отличается от основного
\thanks
    {
   Научный руководитель:  Стрижов~В.\,В.
   Задачу поставил:  Стрижов~В.\,В.
    Консультант:  Исаченко~Р.}
\email
    {alina.samokhina@phystech.edu}
\organization
    {$^1$Московский физико-технический институт(ГУ)}
\abstract
    { {\textbf{Аннотация:}} Данная статья посвящена методам прогнозирования движения с помощью сигналов электрокортикограммы головного мозга.  Цель исследования~--- проверить гипотезу о наличии взаимосвязи между сигналами мозга и движением. Рассматриваются различные методы генерации признаков. Предложен метод снижения размерности исходного признакового пространства с помощью локальных моделей. Пространство параметров модели используется в качестве нового порстранства признаков. Предлагаемое признаковое пространство позволяет обоснованно использовать данные электрокортикограмм при построении моделей нейрокомпьютерного взаимодействия.

\bigskip
\textbf{Ключевые слова}: \emph {электрокортикограмма, нейрокомпьютерный интерфейс}.}

\titleEng
    {Research on the properties of local models in spatial decoding of the brain signals}
\authorEng
    {Samokhina~A.\,M.$^1$, Bolobolova~N.\,A.$^1$, Shiyanov~V.\,A.$^1$}
\organizationEng
    {$^1$ Moscow institute of Physics and Technology (SU)}
\abstractEng
    { \textbf{Abstract:} This paper is devoted to the methods of ECoG signal processing and predicting motion using the results. The main purpose of the research is to show that changes in areas of brain activity is an informative feature for BCI modelling. To see the link between brain signals and motion, we look at different types of feature engineering and compare them. The new  feature space will allow to use ECoG data for building BCI, using ECoG data.

    \bigskip
    \textbf{Keywords}: \emph{ECoG, BCI}.}
    
\begin{document}
\maketitle
\bigskip
\bigskip
\bigskip
\bigskip
\maketitleSecondary
%\linenumbers

\section{Введение}
Нейрокомпьютерный интерфейс позволяет декодировать активность головного мозга для использования внешними устройствами. Это позволяет создавать механизированные конечности, управляемые сигналами головного мозга \cite{Donoghue2008}. В связи с этим в последнее время большое количество работ посвящено методам считывания мозговой активности и декодирования полученной информации \cite{Hu2018}\cite{Song2017}\cite{Loza2017}\cite{Eliseyev2016}\cite{Gaglianese2016}\cite{Bundy2016}\cite{Morishita2014}.
В этой работе используются данные сигналов, полученных инвазивным методом электрокортикографии \cite{Sirven2014}. Сложность декодирования заключается в избыточной размерности сигнала: в данном случае модель является неустойчивой. Построение систем нейрокомпьютерного интерфейса подразумевает использование простых и устойчивых моделей. 

Стандартные подходы состоят в извлечении информативных признаков из пространственных, частотных и временных данных\cite{Morishita2014}\cite{Alexander2013}. Большинство методов исследуют частотные характеристики сигналов\cite{Chin2007}. Подходы \cite{Eliseyev2016}\cite{Motrenko2018} рассматривают все признаки вне зависимости от их природы. Наиболее распространёнными моделями считаются алгоритм PLS\cite{Rosipal2006}\cite{Eliseyev2016}, PCA\cite{Zhao2010}. В работе \cite{Zhao2014} используются алгоритмы, построенные на скрытых марковских моделях, а в  работах \cite{Loza2017}\cite{Song2017} авторы рассматривают различные участки сигнала как слова. 

В данной работе для моделирования фронта распределения сигнала предлагается использовать локальную модель. Это позволяет значительно сократить количество параметров финальной модели, используя пространственную структуру данных. Движение фронта возбуждения приближается с помощью модели прогнозирования движений. В качестве признаков используются параметры построенной в данной работе локальной модели. На полученном признаковом пространстве можно построить простую и устойчивую прогностическую модель.


Полученный  метод значительно снижает размерность данных,  использует пространственную информацию и сохраняет свойства распространения сигнала.



\section{Постановка задачи} Данные электрокортикограммы представляют собой отрезки многомерных временных рядов $\mathbf{X} \in \mathbb{R}^{N \times T}$, где $N$ является числом каналов, с которых получены измерения напряжения, $T$ – параметр времени. Цель работы – построение признакового пространства, облегчающего предсказание по этим данным отклика сигнала – координат конечности $\mathbf{y}(t) \in \mathbb{R}^{3 \times T}$. Конечная модель – $f: \mathbf{X} \to \mathbf{Y}$.


Матрица признакового описания объектов обучающей выборки имеет вид:
\begin{equation}
	\mathbf{X}=
	\begin{pmatrix}
		x_{11}&...&x_{1T}\\
		&...&\\
		x_{N1}&...&x_{NT}
	\end{pmatrix}.
\end{equation}
$\mathbf{y}=(\mathbf{y_1},...,\mathbf{y_n})\T$ матрица ответов обучающей выборки (координат перемещения руки):
\begin{equation}
	\mathbf{y}=
	\begin{pmatrix}
		y_{11}&y_{12}&y_{13}\\
		&...&\\
		y_{n1}&y_{n2}&y_{n3}
	\end{pmatrix},
\end{equation} 
Значение $y_{ij}$ отвечает $j$-й координате траектории движения конечности, соответствующей объекту с признаковым описанием $\mathbf{x}_i}$. 

$\mathbf{X}$ имеет большую размерность и содержит зависимые данные, следовательно модель $f$ оказывается неустойчивой. В работе предлагается рассмотреть локальную модель. Параметрами этой локальной модели будем считать признаковое описание сигнала. Искать параметры будем в виде решения задачи линейной авторегрессии,где необходимо предсказать первый столбец с помощью остальных. Матрица задачи авторегрессии:

\begin{equation}
	\mathbf{x}|\mathbf{Y}=
	\begin{pmatrix}
		x_{t+1}&x_t&...&x_{t-n}\\
		x_t&x_{t-1}&...&x_{t-n-1}\\
		&&...&\\
		x_{t-n}&&...&
	\end{pmatrix}
\end{equation} 

Таким образом, модель $f$ представима в виде композиции моделей $g: \mathbf{X} \to \mathbf{w}$ и $h: \mathbf{w} \to \mathbf{Y}$. Целевые параметры $\mathbf{w}$ находятся путем минимизации функции ошибки $L$:
\begin{equation}
	\mathbf{w^*} = \argmin L(\mathbf{X}, \mathbf{y}, \mathbf{w}, g, h).
\end{equation}

Цель работы состоит в нахождении оптимальной модели $g$ для получения признакового пространства $\mathbf{w^*}$



\bibliographystyle{plain}
\bibliography{Samokhina2018Project17}


\end{document}
