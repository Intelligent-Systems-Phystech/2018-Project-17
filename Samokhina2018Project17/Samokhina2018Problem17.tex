\documentclass[12pt,twoside]{article}
\usepackage{jmlda}

%\NOREVIEWERNOTES
\title
    [Исследование свойств локальных моделей при пространственном декодировании сигналов головного мозга] 
    {Исследование свойств локальных моделей при пространственном декодировании сигналов головного мозга}
\author
    [Самохина~А.\,М.] % список авторов для колонтитула; не нужен, если основной список влезает в колонтитул
    {Самохина~А.\,М.$^1$, Болоболова~Н.\,А.$^1$, Шиянов~В.\,A.$^1$} % основной список авторов, выводимый в оглавление
    %[Автор~И.\,О.$^1$, Соавтор~И.\,О.$^2$, Фамилия~И.\,О.$^2$] % список авторов, выводимый в заголовок; не нужен, если он не отличается от основного
\thanks
    {
   Научный руководитель:  Стрижов~В.\,В.
   Задачу поставил:  Стрижов~В.\,В.
    Консультант:  Исаченко~Р.}
\email
    {alina.samokhina@phystech.edu}
\organization
    {$^1$Московский физико-технический институт(ГУ)}
\abstract
    { {\textbf{Аннотация:}} Данная статья посвящена методам прогнозирования движения с помощью сигналов электрокортикограммы головного мозга.  Основная задача исследования~--- показать, что наблюдаемое изменение зон активности мозга является информативным признаком для построения нейрокомпьютерных интерфейсов. Чтобы проследить связь между сигналами мозга и движением, в данном исследовании рассматриваются различные методы генерации признаков. Предлагаемое признаковое пространство позволяет обоснованно использовать данные электрокортикограмм при построении моделей нейрокомпьютерного взаимодействия.

\bigskip
\textbf{Ключевые слова}: \emph {электрокортикограмма, нейрокомпьютерный интерфейс}.}

\titleEng
    {Research on the properties of local models in spatial decoding of the brain signals}
\authorEng
    {Samokhina~A.\,M.$^1$, Bolobolova~N.\,A.$^1$, Shiyanov~V.\,A.$^1$}
\organizationEng
    {$^1$ Moscow institute of Physics and Technology (SU)}
\abstractEng
    { \textbf{Abstract:} This paper is devoted to the methods of ECoG signal processing and predicting motion using the results. The main purpose of the research is to show that changes in areas of brain activity is an informative feature for BCI modelling. To see the link between brain signals and motion, we look at different types of feature engineering and compare them. The new  feature space will allow to use ECoG data for building BCI, using ECoG data.

    \bigskip
    \textbf{Keywords}: \emph{ECoG, BCI}.}
    
\begin{document}
\maketitle
\bigskip
\bigskip
\bigskip
\bigskip
\maketitleSecondary
%\linenumbers

\section{Введение}
Нейрокомпьютерные интерфейсы позволяют превращать активность головного мозга в сигналы для внешних устройств. Это позволяет существенно улучшить качество жизни людей с серьёзными нарушениями работы двигательного аппарата \cite{Donoghue2008}, в связи с чем в последнее время большое количество работ связано с методами считывания мозговой активности и декодирования полученной информации \cite{Hu2018}\cite{Song2017}\cite{Loza2017}\cite{Eliseyev2016}\cite{Gaglianese2016}\cite{Bundy2016}\cite{Morishita2014}, и другие. 
В этой работе мы использовали данные сигналов, полученных инвазивным методом электрокортикограммы \cite{Sirven2014}. Одной из главных проблем декодирования этих сигналов и превращения их в модели для нейрокомпьютерных интерфейсов является объём. В среднем размерность признакового пространства составляет порядка $10^3$ признаков. В данном случае модель является неустойчивой, в то время как построение систем нейрокомпьютерного интерфейса подразумевает использование моделей простых и устойчивых. Таким образом, важным этапом построения нейрокомпьютерного взаимодействия является построение нового адекватного признакового пространства. 
Стандартные подходы состоят в извлечении информативных признаков из пространственных, частотных и временных данных. Большинство методов исследуют только некоторую часть данных, чаще всего, из частотную составляющую, выделяя спектр. Однако в недавнее время стали появляться подходы, основанные на нескольких признаках и позволяют рассматривать признаки вне зависимости от их природы \cite{Eliseyev2016}\cite{Motrenko2018}. Для построения признакового пространства и модели используется большое количество методов.  Наиболее распространёнными считаются алгоритм PLS\cite{Rosipal2006}\cite{Eliseyev2016}, PCA\cite{Zhao2010}. Также можно встретить алгоритмы, использующие HMM\cite{Zhao2014} или совершенно иные\cite{Loza2017} методы. 
Мы, в свою очередь, использовали локальную модель, что позволило значительно сократить количество признаков, сохранив структуру данных. Описав движение фронта возбуждения несколькими векторами, мы приблизили сигнал локальной моделью, взяв в качестве новых признаков её параметры. Таким образом, наша команда сформировала признаковое пространство и локально смоделировала пространственный сигнал, на основе которого можно построить устойчивую, простую и адекватную прогностическую модель.
Полученный  метод выигрывает у уже известных за счёт того, что значительно снижает размерность данных,  использует пространственную информацию и принимает во внимание свойства распространения сигнала. Однако сложно не заметить, что во избежание смещения модели мы были вынуждены наложить строгое ограничение на вид при выборе семейства локальных моделей. Над этим мы планируем продолжить работу в дальнейших исследованиях.

\bibliographystyle{plain}
\bibliography{Samokhina2018Project17}


\end{document}
