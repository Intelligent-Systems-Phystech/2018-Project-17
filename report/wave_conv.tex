\section{Преобразование данных}
Свертка - это процедура, в которой два сигнала объединяются для получения третьего сигнала, который имеет общие характеристики двух входных сигналов. 
На языке свертки два входных временных ряда называются сигналом и ядром.
Интересующий нас временной ряд (например, данные ECoG) мы называем сигналом, a фильтр ядром.
При анализе данных ECoG свертка используется для выделения активности, специфичной для полосы частот, и для локализации этой активности во времени. Это делается с помощью ограниченных по времени синусоид – вейвлетов. Поскольку вейвлет (ядро свертки) перетаскивается вдоль данных ECoG (сигнал свертки), он показывает, когда и в какой степени данные ECoG содержат элементы, которые выглядят как вейвлет. Когда вейвлет-свертка повторяется для одних и тех же данных ECoG с использованием вейвлетов разных частот, может быть сформировано частотно-временное представление. 
Преимущество вейвлет-преобразования перед преобразованием Фурье состоит в том, что оно позволяет проследить за изменением спектральных свойств сигнала со временем и
указать, какие частоты доминируют в сигнале.

