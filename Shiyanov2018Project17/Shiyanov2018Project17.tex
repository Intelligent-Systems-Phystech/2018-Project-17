\documentclass[12pt,twoside]{article}
\usepackage{../jmlda}

%\NOREVIEWERNOTES
\title
    [Локальные модели при декодировании сигналов головного мозга]
    {Исследование свойств локальных моделей при пространственном декодировании сигналов головного мозга.}
\author
    [Шиянов~В.\,А.]
    {Шиянов~В.\,А., Болоболова~Н.\,А., Самохина~А.\,М., Мокруполо~М.\,Н.}
\thanks
    {Работа выполнена при финансовой поддержке РФФИ, проект \No\,00-00-00000.
    Научный руководитель:  Стрижов~В.\,В.
    Задачу поставил: Стрижов~В.\,В.
    Консультант: Исаченко~Р.\,О.}
\email
    {vadimsh@phystech.edu}
%\organization
%    {$^1$Организация; $^2$Организация}
\abstract
    {Целью работы является восстановление связи между сигналами электрокортикограммы и пространственным движением конечностей тела.
    Особенностью является избыточность данных кортикограммы. Это позволяет снизить размерность задачи. В данном исследовании предлагается использовать пространственную информацию, то есть перемещение зон активности мозга. Для этого предлагается построить локальную модель описания сигнала и использовать ее параметры в качестве признакового описания.
    С помощью полученных признаков предлагается обучить алгоритм машинного обучения, который позволил бы предсказывать движения конечностей по сигналам головного мозга.
    В качестве данных предлагается использовать данные электрокортикограмм обезьян и движения их конечностей.

    \bigskip
    \textbf{Ключевые слова}: \emph{Brain-Computer Interface, feature engineering}.}
\begin{document}
\maketitle
\section{Введение}
Целью данной работы является построение модели, которая смогла бы связать сигналы мозга с движениями конечностей тела. Предлагается использовать пространственную составляющую сигнала, то есть перемещение зон активности головного мозга. Сложностью исследования является избыточное, высокоразмерное пространство сигналов, которое приводит к неустойчивой модели.\\
Для борьбы с высокой размерностью пространства признаков предлагается использовать ряд алгоритмов. В том числе PCA \cite{Jolliffe2011} для выделения наиболее важных признаков, то есть признаков, ответственных за наибольшую часть отклонения в выборке. Также предлагается использовать алгоритм PLS \cite{Haenlein2004} чтобы учесть латентную природу связей между сигналами головного мозга и движением тела. Также предлагается использовать алгоритм CCA \cite{thompson2005canonical} для выбора наиболее связанных из двух наборов переменных. Наиболее близкой к данной работе, является работа \cite{Motrenko_2018}. Эта работа также посвящена декодированию сигналов головного мозга и движения конечностей, однако, авторы ограничиваются частотными характеристиками сигналов.\\
Данное исследование предлагает использовать пространственную структуру сигнала. Для уменьшения размерности задачи предлагается построить локальную модель описания сигнала. Параметры этой модели предлагается использовать в качестве признакового описания сигнала. Такой подход позволяет заметно снизить размерность итоговых признаков, то есть позволяют построить более простую и устойчивую модель. Однако, результат построения локальной модели сильно зависит от изначального выбора признакового пространства, что влечет за собой ограничения на вид возмущения.\\
В работе предлагается использовать данные neurotycho (\url{http://neurotycho.org}), которые представляют из себя кортикограммы обезьян и движения их конечностей, записанные одновременно. С помощью этих данных предлагается обучить модель и проверить ее устойчивость и точность.

%\linenumbers
\bibliographystyle{plain}
\bibliography{../Project17}

\end{document}
