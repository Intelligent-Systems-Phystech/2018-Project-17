\documentclass[12pt,twoside]{article}
\usepackage{jmlda}
\usepackage{mathtools}
%\NOREVIEWERNOTES
\title
    [Прогнозирование намерений]
    {Прогнозирование намерений. Исследование свойств локальных моделей при пространственном декодировании сигналов головного мозга.}
\author
    [Болоболова~Н.\,А.]
    {Шиянов~В.\,А., Болоболова~Н.\,А., Самохина~А.\,М., Мокруполо~М.\,Н.}
\thanks
    {Работа выполнена при финансовой поддержке РФФИ, проект \No\,00-00-00000.
    Научный руководитель:  Стрижов~В.\,В.
    Задачу поставил: Стрижов~В.\,В.
    Консультант: Исаченко~Р.\,О.}
\email
    {nataboll@mail.ru}
%\organization
%    {$^1$Организация; $^2$Организация}
\abstract
    {В данной работе исследуются механизмы регуляции движения конечностей нейронами головного мозга. Проверяется гипотеза о влиянии перемещения зон активности нейронов на траекторию движения конечности. Высокая размерность признакового пространства сигналов приводит к неустойчивости модели. Предполагается, что размерность пространства можно уменьшить. Сигнал высокой размерности аппроксимируется локальной моделью. Пространство параметров модели используется как признаковое пространство. Таким образом, модель упрощается и становится устойчивее. В задаче используются данные электрокортикограмм, собранные на основе исследований активности нейронов головного мозга обезьян.    	

\bigskip
\textbf{Ключевые слова}: \emph{Brain-Computer Interface (BCI), feature engineering}.}
\titleEng
    {Spatial Signal Decoding for Brain Computer Interface.}
\authorEng
    {Shiyanov~V.\,A., Bolobolova~N.\,A., Samokhina~A.\,M., Mokrypolo~M.\,N.}
%\organizationEng
%    {$^1$Organization; $^2$Organization}
\abstractEng
    {The project is mainly aimed at understanding neural regulation of limb movement. To build dependencies, one needs a significant description of brain signals and related behavioral responses. This work is focused on building a model (objects, features), based on time series obtained from the results of the neural activity research (the main source is ECoG). The model is expected to help in testing the hypothesis that the movement of brain activity areas is an informative feature for solving the BCI problem. In earlier studies, the frequencies present in the signal of the electrocorticogram were used to predict. Here frequencies are suggested to be replaced with another set of features, using spatial structure of signals, taken by sensors. Data used here have been collected using monkeys' brain neural activity and their behavior dependencies studies under various conditions. Dataset is submitted on Neurotycho.org.

\bigskip
\textbf{Keywords}: \emph{Brain-Computer Interface (BCI), feature engineering}.}
\begin{document}
\maketitle
\bigskip
\bigskip
\bigskip
\bigskip
\bigskip
\maketitleSecondary

\section{Введение}
Нейрокомпьютерный интерфейс (Brain Computer Interface) позволяет восстановить мобильность людей с нарушениями двигательных функций. Алгоритм BCI с высокой точностью транслирует сигналы нейронов головного мозга в команды для исполняющей системы. Это дает возможность регулировать движение роботизированной конечности в соответствии с механизмами нервной регуляции. Исследование состоит в поиске зависимостей между сигналами ECoG и движениями конечностей. Целью является точное предсказание траектории движения в трехмерном пространстве. В естественной среде на сигналы моторных областей накладываются сторонние шумы: импульсы других долей головного мозга и сигналы из внешней среды. Огромная размерность пространства сигналов приводит к переобучению и нестабильности алгоритма. \\\\
Работа \cite{Eliseyev2014} посвящена построению модели на основе таких характеристик сигналов, как частота, амплитуда и временная локализация, однако результат недостаточно устойчив по отношению к шумовым сигналам. В исследовании \cite{Song2017} рассмотрены механизмы пространственной фильтрации сигналов, уменьшения размерности задачи с использованием метода главных компонент и кластеризации данных методом Blind Source Separation, но полностью от шумовых сигналов избавиться не удалось из-за больших вариаций амплитуд. В работе \cite{Motrenko_2018} исследован метод отбора признаков посредством решения задачи квадратичного программирования (Quadratic Programming Feature Selection \cite{rodriguez2010quadratic}). \\\\
В данной работе предлагается использовать локальную структуру сигналов. Движение фронта сигнала имеет пространственную структуру, следовательно, количество параметров модели значительно уменьшается. Получается более простая аппроксимация сигнала высокой размерности, и, в то же время, более устойчивая модель. Параметры полученной локальной модели используются в качестве признакового описания объекта. В смежных исследованиях признаки строятся только на основе частотных характеристик. \\\\
В эксперименте используются данные из ресурса \url{http://neurotycho.org/}. Сбор данных производился с использованием методики Multi-Dimensional Recording (\url{http://wiki.neurotycho.org/Multi-dimensional_Recording}). Запись сигналов ECoG и траектории движения руки проводилась одновременно. Каждый из экспериментов длился 15 минут, первые 8 минут производилась запись обучающей выборки, а оставшиеся 7 минут - запись тестовой выборки. \\\\

\section{Постановка задачи}
В качестве базового объекта используются отрезки временных рядов $s_i$, $i=1,...,n$. Обозначим $\mathbf{s}=(s_1,...,s_n)\T$. Для получения признаков из исходных данных будем использовать локальную модель $g(\mathbf{s}, \mathbf{\Theta})$, описывающую пространственную структуру сигнала. В этих обозначениях $\mathbf{\Theta}$ - параметры порождающей локальной модели, используемые для построения признакового описания объекта $\mathbf{s}$:
\begin{equation}
 \mathbf{x_i} = \mathbf{\Theta}(s_i).
\end{equation}
Таким образом, получаем матрицу признакового описания объектов обучающей выборки:
\begin{equation}
	\mathbf{X}=
	\begin{pmatrix}
		\mathbf{x_1}\T\\
		.\\
		.\\
		.\\
		\mathbf{x_n}\T
	\end{pmatrix}.
\end{equation}
Обозначим $\mathbf{y}=(\mathbf{y_1},...,\mathbf{y_n})\T$ матрицу ответов обучающей выборки (координат перемещения руки):
\begin{equation}
	\mathbf{y}=
	\begin{pmatrix}
		y_{11}&y_{12}&y_{13}\\
		&...&\\
		y_{n1}&y_{n2}&y_{n3}
	\end{pmatrix},
\end{equation} 
где значение $y_{ij}$ отвечает $j$-й координате траектории движения конечности, соответствующей объекту с признаковым описанием $\mathbf{x_i}$. Параметры локальной модели можно получать, решая задачу авторегрессии с матрицей:
\begin{equation}
	\mathbf{x}|\mathbf{Y}=
	\begin{pmatrix}
		x_{t+1}&x_t&...&x_{t-n}\\
		x_t&x_{t-1}&...&x_{t-n-1}\\
		&&...&\\
		x_{t-n}&&...&
	\end{pmatrix}
\end{equation} 
(необходимо предсказать первый столбец, используя все остальные, получив параметры регрессии $\mathbf{\Theta}$).
После построения признакового описания объектов обучается регрессионная модель $f$. Целевые параметры $\mathbf{w}$ находятся путем минимизации функции ошибки $L$:
\begin{equation}
	\mathbf{w^*} = \argmin L(\mathbf{X}, \mathbf{y}, \mathbf{w}, f).
\end{equation}
Цель состоит в отыскании оптимальной локальной модели $g$ для получения признакового пространства.
%\linenumbers
\bibliographystyle{plain}
\bibliography{../Project17}

\end{document}
