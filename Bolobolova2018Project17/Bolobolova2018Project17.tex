\documentclass[12pt,twoside]{article}
\usepackage{jmlda}
\usepackage{mathtools}
\usepackage{lineno}
\usepackage{setspace}
\linenumbers
\doublespacing
%\NOREVIEWERNOTES
\title
    [Прогнозирование намерений]
    {Прогнозирование намерений. Исследование свойств локальных моделей при пространственном декодировании сигналов головного мозга.}
\author
    [Болоболова~Н.\,А.]
    {Шиянов~В.\,А., Болоболова~Н.\,А., Самохина~А.\,М., Мокруполо~М.\,Н.}
\thanks
    {Работа выполнена при финансовой поддержке РФФИ, проект \No\,00-00-00000.
    Научный руководитель:  Стрижов~В.\,В.
    Задачу поставил: Стрижов~В.\,В.
    Консультант: Исаченко~Р.\,О.}
\email
    {nataboll@mail.ru}
%\organization
%    {$^1$Организация; $^2$Организация}
\abstract
    {В данной работе исследуются механизмы регуляции движения конечностей нейронами головного мозга. Проверяется гипотеза о влиянии перемещения зон активности нейронов на траекторию движения конечности. Высокая размерность признакового пространства сигналов приводит к неустойчивости модели машинного обучения. Сигнал высокой размерности предлагается аппроксимировать локальной моделью, что существенно уменьшает размерность пространства параметров. Пространство параметров локальной модели используется как признаковое пространство. Таким образом, результирующая модель становится проще и устойчивее. В задаче используются данные электрокортикограмм, собранные на основе исследований активности нейронов головного мозга обезьян.    	

\bigskip
\textbf{Ключевые слова}: \emph{Brain-Computer Interface (BCI), feature engineering}.}
\begin{document}
\maketitle
\bigskip
\bigskip
\bigskip
\bigskip
\bigskip
\maketitleSecondary

\section{Введение}
\section{Введение}
Нейрокомпьютерный интерфейс (Brain Computer Interface) \cite{Morishita2014} позволяет восстановить мобильность людей с нарушениями двигательных функций.  Алгоритм BCI транслирует сигналы нейронов головного мозга в команды для исполняющей системы \cite{Morishita2014}. Это дает возможность регулировать движение роботизированной конечности в соответствии с механизмами нервной регуляции. \cite{Donoghue2008}. 


В последнее время большое количество работ посвящено методам считывания мозговой активности и декодирования информации \cite{Hu2018}\cite{Song2017}\cite{Loza2017}\cite{Eliseyev2016}\cite{Gaglianese2016}\cite{Bundy2016}\cite{Morishita2014}.
В этой работе используются данные сигналов, полученных инвазивным методом электрокортикографии (ECoG) \cite{Sirven2014}. Сложность декодирования заключается в избыточной размерности сигнала. Модель прогнозирования намерений является неустойчивой. Также к нестабильности модели приводит наличие сторонних шумов, накладывающихся на сигналы в естественной среде: импульсов других долей головного мозга и сигналов из внешней среды.  Для построения системы нейрокомпьютерного интерфейса необходимо использовать простую и устойчивую модель. 


Исследование состоит в восстановлении зависимостей между сигналами ECoG и движениями конечностей. Для точного предсказания траектории движения в трехмерном пространстве требуется снизить размерность признакового пространства и снизить влияние шумов на предсказания.


Стандартные подходы состоят в извлечении информативных признаков из пространственных, частотных и временных характеристик сигнала\cite{Morishita2014}\cite{Alexander2013}. Большинство методов в смежных работах исследуют частотные характеристики\cite{Chin2007}, \cite{Eliseyev2014}, \cite{Loza2017}. В работах \cite{Eliseyev2016}\cite{Motrenko2018} рассматриваются все признаки вне зависимости от их природы. Наиболее распространёнными моделями являются алгоритмы PLS\cite{Rosipal2006}\cite{Eliseyev2016}, PCA\cite{Zhao2010},\cite{Song2017}. В работах \cite{Zhao2014} используются алгоритмы, построенные на скрытых марковских моделях. В  работах \cite{Loza2017}\cite{Song2017} авторы рассматривают различные участки сигнала как слова. В работе \cite{Motrenko2018} исследован метод отбора признаков с помощью квадратичного программирования (Quadratic Programming Feature Selection \cite{rodriguez2010quadratic}). Результаты работ недостаточно устойчивы по отношению к шумовым сигналам из-за больших вариаций амплитуд \cite{Eliseyev2014},\cite{Song2017}, несмотря на рассмотренные в этих работах механизмы пространственной фильтрации сигналов.


В данной работе для моделирования фронта распределения сигнала предлагается использовать локальную модель. Движение фронта возбуждения приближается с помощью локальной модели прогнозирования движений. В качестве признакового описания объектов используются параметры построенной локальной модели. Полученный  метод значительно снижает размерность данных, использует пространственную информацию и сохраняет свойства распространения сигнала.
Как следствие, количество параметров конечной модели значительно уменьшается. Получается более простая аппроксимация сигнала высокой размерности и более устойчивая прогностическая модель.


В эксперименте используются данные с сайта neurotycho.org. Сбор данных производился с использованием методики Multi-Dimensional Recording. Запись сигналов ECoG и траектории движения руки проводилась одновременно. Каждый из экспериментов длился 15 минут, первые 8 минут производилась запись обучающей выборки, оставшиеся 7 минут - запись тестовой выборки. \\\\


\section{Постановка задачи}
Исходные данные - отрезки многомерных временных рядов электрокортикограммы. Пусть $\mathbf{X} \in \mathbb{R}^{T \times N}$ - матрица значений напряжения, где $N$ – число каналов, $T$ – параметр времени. Необходимо построить информативное признаковое пространство для предсказания траектории движения конечности $\mathbf{y}(t)$. \\
Матрицы обучающей выборки имеют вид:
\begin{equation}
\mathbf{X} = \{x_{ij}\}_{i=1,\dots,T;\ \ j=1,\dots,N};
\end{equation} 
\begin{equation}
\mathbf{Y} = \{y_{ij}\}_{i=1,\dots,T;\ \ j=1,2,3},
\end{equation}
где $\mathbf{X}$ - набор сигналов, $\mathbf{Y}$ - матрица ответов для входных данных $\mathbf{X}$. Объектом будем называть строку $\mathbf{x}_i$ матрицы $\mathbf{X}$. В ней содержатся измерения в момент времени $t_i$, $i = 1,\dots,T$. Строка состоит из $N$ элементов, каждый из которых соответствует каналу. Значение $y_{ij}$ отвечает $j$-й координате траектории движения конечности, соответствующей объекту $\mathbf{x}_i$.\\
Итоговую модель прогнозирования $f:\mathbf{X}\to\mathbf{Y}'$ предлагается представить в виде композиции двух моделей: $f(\mathbf{X})=h(\mathbf{w}, g(\mathbf{\Theta}, \mathbf{X}))$. Здесь $\mathbf{Y}'$ - предсказания модели $f$ для обучающей выборки $\mathbf{X}$. С помощью модели $g(\mathbf{\Theta}, \mathbf{X})$ строится новое описание данных - признаковое пространство $\mathbf{W}$: $g:\mathbf{X}\to\mathbf{W}$. Так как модель $g(\mathbf{\Theta}, \mathbf{X})$ использует локальную пространственную структуру сигнала для аппроксимации перемещения фронта возбуждения, будем называть ее \textit{локальной моделью}.
 Множество $\mathbf{\Theta}$ - это \textit{параметры локальной модели}, которые предлагается использовать для построения признакового описания объектов. Параметры $\mathbf{\Theta}$ могут быть найдены решением задачи авторегрессии с матрицей:
\begin{equation}
\left(\begin{array}{@{}c|ccc@{}}
x_{i, t+1} & x_{i, t}   & \cdots & x_{i, t-n}   \\
x_{i, t}   & x_{i, t-1} & \cdots & x_{i, t-n-1} \\
\cdots     & \cdots     & \cdots & \cdots
\end{array}\right)
\begin{pmatrix}
\theta_1 \\
\theta_2 \\
\cdots
\end{pmatrix}.
\end{equation}
Модель $h(\mathbf{w}, \mathbf{W}): \mathbf{W}\to\mathbf{Y}'$ является моделью линейной регрессии с параметрами $\mathbf{w}$. На этапе применения модели $h(\mathbf{w}, \mathbf{W})$ построенное признаковое описание $\mathbf{W}$ используется для предсказания траекторий $\mathbf{Y}'$.
Параметры $\mathbf{w}$ модели $h(\mathbf{w}, \mathbf{W})$ находятся путем минимизации функции потерь $L(\mathbf{X}, \mathbf{Y}, \mathbf{Y}', \mathbf{w}, g, h)$:
\begin{equation}
\mathbf{w^*} = \argmin_{\mathbf{w}} L(\mathbf{X}, \mathbf{Y}, \mathbf{Y}', \mathbf{w}, g, h).
\end{equation}
В качестве функции потерь можно выбрать, например, квадратичную ошибку:
\begin{equation}
L(\mathbf{Y}, \mathbf{Y}') = \sum_{j=1}^{3}\|\mathbf{Y}_j-\mathbf{Y}_j'\|^2_2,
\end{equation}
где $\mathbf{Y}_j$ и $\mathbf{Y}_j'$ - $j$-е столбцы матрицы ответов $\mathbf{Y}$ и матрицы предсказаний $\mathbf{Y}'$ соответственно.
Цель работы состоит в нахождении оптимальной локальной модели $g(\mathbf{\Theta}, \mathbf{X})$ для построения информативного признакового пространства.
%\linenumbers
\bibliographystyle{plain}
\bibliography{../Project17}

\end{document}
