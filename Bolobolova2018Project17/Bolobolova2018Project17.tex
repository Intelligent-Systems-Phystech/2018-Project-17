\documentclass[12pt,twoside]{article}
\usepackage{jmlda}
\usepackage{mathtools}
\usepackage{lineno}
\usepackage{setspace}
\linenumbers
\doublespacing
%\NOREVIEWERNOTES
\title
    [Прогнозирование намерений]
    {Прогнозирование намерений. Исследование свойств локальных моделей при пространственном декодировании сигналов головного мозга.}
\author
    [Болоболова~Н.\,А.]
    {Шиянов~В.\,А., Болоболова~Н.\,А., Самохина~А.\,М., Мокруполо~М.\,Н.}
\thanks
    {Работа выполнена при финансовой поддержке РФФИ, проект \No\,00-00-00000.
    Научный руководитель:  Стрижов~В.\,В.
    Задачу поставил: Стрижов~В.\,В.
    Консультант: Исаченко~Р.\,О.}
\email
    {nataboll@mail.ru}
%\organization
%    {$^1$Организация; $^2$Организация}
\abstract
    {В данной работе исследуются механизмы регуляции движения конечностей нейронами головного мозга. Проверяется гипотеза о влиянии перемещения зон активности нейронов на траекторию движения конечности. Высокая размерность признакового пространства сигналов приводит к неустойчивости модели машинного обучения. Сигнал высокой размерности предлагается аппроксимировать локальной моделью, что существенно уменьшает размерность пространства параметров. Пространство параметров локальной модели используется как признаковое пространство. Таким образом, результирующая модель становится проще и устойчивее. В задаче используются данные электрокортикограмм, собранные на основе исследований активности нейронов головного мозга обезьян.    	

\bigskip
\textbf{Ключевые слова}: \emph{Brain-Computer Interface (BCI), feature engineering}.}
\begin{document}
\maketitle
\bigskip
\bigskip
\bigskip
\bigskip
\bigskip
\maketitleSecondary

\section{Введение}
Нейрокомпьютерный интерфейс (Brain Computer Interface) \cite{Morishita2014} позволяет восстановить мобильность людей с нарушениями двигательных функций. Алгоритм BCI транслирует сигналы нейронов головного мозга в команды для исполняющей системы \cite{Morishita2014}. Это дает возможность регулировать движение роботизированной конечности в соответствии с механизмами нервной регуляции. Исследование состоит в восстановлении зависимостей между сигналами ECoG (electrocorticogram) и движениями конечностей. Для точного предсказания траектории движения в трехмерном пространстве требуется снизить размерность признакового пространства. В естественной среде на сигналы моторных областей накладываются сторонние шумы: импульсы других долей головного мозга и сигналы из внешней среды. Огромная размерность пространства сигналов приводит к переобучению и нестабильности алгоритма. \\\\
Работа \cite{Eliseyev2014} посвящена построению модели на основе таких характеристик сигналов, как частота, амплитуда и временная локализация, но результат недостаточно устойчив по отношению к шумовым сигналам. В исследовании \cite{Song2017} рассмотрены механизмы пространственной фильтрации сигналов, снижения размерности задачи с использованием метода главных компонент и кластеризации данных методом Blind Source Separation, но полностью от шумовых сигналов избавиться не удалось из-за больших вариаций амплитуд. В работе \cite{Motrenko_2018} исследован метод отбора признаков с помощью квадратичного программирования (Quadratic Programming Feature Selection \cite{rodriguez2010quadratic}). \\\\
В данной работе предлагается использовать локальную структуру сигналов. Движение фронта сигнала имеет пространственную структуру и задается небольшим количеством параметров. Как следствие, при использовании характеристик этой структуры количество параметров модели значительно уменьшается. Получается более простая аппроксимация сигнала высокой размерности, но более устойчивая модель. Параметры локальной модели используются в качестве признакового описания объекта. В смежных исследованиях \cite{Eliseyev2014}, \cite{Loza2017} признаки строятся только на основе частотных характеристик. \\\\
В эксперименте используются данные с сайта \url{http://neurotycho.org/}. Сбор данных производился с использованием методики Multi-Dimensional Recording (\url{http://wiki.neurotycho.org/Multi-dimensional_Recording}). Запись сигналов ECoG и траектории движения руки проводилась одновременно. Каждый из экспериментов длился 15 минут, первые 8 минут производилась запись обучающей выборки, оставшиеся 7 минут - запись тестовой выборки. \\\\

\section{Постановка задачи}
Данные электрокортикограммы представляют собой отрезки многомерных временных рядов $\mathbf{X} \in \mathbb{R}^{T \times N}$, где $N$ – число каналов, с которых получены значения напряжения, $T$ – параметр времени. Необходимо построить признаковое пространство для предсказания траектории движения конечности $\mathbf{y}(t) \in \mathbb{R}^{T \times 3}$. \\
Матрицы обучающей выборки имеют вид:
\begin{equation}
\mathbf{X}=
\begin{pmatrix}
x_{11}&...&x_{1N}\\
&...&\\
x_{T1}&...&x_{TN}
\end{pmatrix},\ \
\mathbf{Y}=
\begin{pmatrix}
y_{11}&y_{12}&y_{13}\\
&...&\\
y_{T1}&y_{T2}&y_{T3}
\end{pmatrix}.
\end{equation} 
Значение $y_{ij}$ отвечает $j$-й координате траектории движения конечности, соответствующей объекту $\mathbf{x}_i$.\\
Итоговая модель $f:\mathbf{X}\to\mathbf{Y}$ представима в виде композиции моделей $g: \mathbf{X} \to \mathbf{W}$ и $h: \mathbf{W}\times\mathbf{\Theta} \to \mathbf{Y}$, где $\mathbf{W}$ - признаковое пространство, $\mathbf{\Theta}$ - параметры порождающей локальной модели $g(\mathbf{X}, \mathbf{\Theta})$. Локальная модель $g(\mathbf{X}, \mathbf{\Theta})$ описывает пространственную структуру сигнала. Параметры $\mathbf{\Theta}$ являются решением задачи авторегрессии с матрицей:
\begin{equation}
\left(\begin{array}{@{}c|ccc@{}}
x_{i, t+1} & x_{i, t}   & \cdots & x_{i, t-n}   \\
x_{i, t}   & x_{i, t-1} & \cdots & x_{i, t-n-1} \\
\cdots     & \cdots     & \cdots & \cdots
\end{array}\right)
\begin{pmatrix}
\theta_1 \\
\theta_2 \\
\cdots
\end{pmatrix}.
\end{equation}
Параметры $\mathbf{w}$ регрессионной модели прогнозирования $h(\mathbf{W}, \mathbf{\Theta}, \mathbf{w})$ находятся путем минимизации функции ошибки $L(\mathbf{X}, \mathbf{y}, \mathbf{w}, g, h)$:
\begin{equation}
\mathbf{w^*} = \argmin_{\mathbf{w}} L(\mathbf{X}, \mathbf{y}, \mathbf{w}, g, h).
\end{equation}
Цель работы состоит в нахождении оптимальной локальной модели $g(\mathbf{X}, \mathbf{\Theta})$ для получения признакового пространства.
%\linenumbers
\bibliographystyle{plain}
\bibliography{../Project17}

\end{document}
