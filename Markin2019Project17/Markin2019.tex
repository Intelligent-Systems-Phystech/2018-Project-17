\documentclass[12pt,twoside]{article}


\usepackage{graphicx}
\usepackage{caption}
%\usepackage[russian]{babel}
\usepackage{jmlda}
\usepackage{lineno}
\usepackage{setspace}
\usepackage{lineno}
\linenumbers

\usepackage{setspace}
\doublespacing

\title{
Локальные модели при декодировании сигналов головного мозга}
\author{Маркин~В.\,О}
\date{201}

\email
    {markin.vo@phystech.edu}


\organization
    {Московский физико-технический институт}
\abstract
	{В работе рассматривается задача построения оптимального признакового описания в задаче декодирования сигналов. Рассматриваются электрические сигналы в коре головного мозга, записанные при помощи электрокортикографии (ECoG). Исходное признаковое пространство избыточно, модель прогнозирования оказывается неустойчивой. Для решения данной проблемы предлагается построить локальную модель аппроксимации сигнала. Это позволяет существенно снизить размерность признакового пространства и учесть пространственную структуру сигнала. В статье приведены результаты численных экспериментов на данных электрокортикограмм головного мозга обезьян. Проводится сравнение различных методов отбора признаков и гипотез порождения данных

\bigskip
\textbf{Ключевые слова}: \emph {Локальные модели, отбор признаков, нейрокомпьютерный интерфейс}

}






\begin{document}
\bigskip
\bigskip\bigskip
\maketitle
\bigskip
\section{Введение}
Нейрокомпьютерный интерфейс (BCI) позволяет считывать сигналы нейронов головного мозга и декодировать их в команды исполняющей системы. Исследования в данной области позволяют восстанавливать дееспособность людей с нарушениями двигательных функций организма. Примером такой системы является система управления роботизированным протезом посредством мозговых импульсов. 

Мозговая активность представляет собой совокупность электрических импульсов различной амплитуды и частоты, возникающих в коре головного мозга. Электроды, закрепленные в коре, позволяют считывать сигналы для их дальнейшего декодирования алгоритмами нейрокомпьютерного интерфейса.

Стандартные подходы к решению задачи состоят в извлечении информативных признаков из пространственных, частотных и временных характеристик сигнала~\cite{Morishita2014,Alexander2013}. Большинство методов в смежных работах исследуют частотные характеристики~\cite{Chin2007,Eliseyev2014,Loza2017}. Наиболее распространёнными моделями являются алгоритмы PLS~\cite{Rosipal2006,Eliseyev2016,Eliseyev2014}, PCA~\cite{Zhao2010,Song2017}. В работе~\cite{Zhao2014} используются алгоритмы, построенные на скрытых марковских моделях. В  работах~\cite{Loza2017,Song2017} авторы рассматривают различные участки сигнала в виде слов. В работе~\cite{Strijov2018} задача отбора признаков сводится к задаче квадратичного программирования (Quadratic Programming Feature Selection~\cite{rodriguez2010quadratic}).
Также для решения задачи используются нейросетевые модели\cite{Xie2018DeepLA}. В этой работе для извлечения признаков используются свреточная нейронная сеть, а для пренсказания - сеть LSTM.

В данной работе для учета пространственной структуры сигнала предлагается построить локальную модель аппроксимации сигнала, поступающего от электродов. Параметры полученной локальной модели используются в качестве нового признакового описания. Данный подход позволяет снизить размерность пространства признаков и повысить устойчивость модели. Для достижения лучшего качества предлагается использовать приемы, приведенные в работе~\cite{Bundy2016}. В данной статье предлагается перед непосредственным предсказанием траектории движения кисти определить движется ли она в данный момент или нет. Предлагается предсказывать траекторию руки и ее скорость, так как скорость сильнее кореллирована с импульсами, чем координата.

В вычислительном эксперименте используются данные электрокортикограмм обезьян с сайта neurotycho.org.


\bibliography{Markin}
\bibliographystyle{unsrt}
\end{document}